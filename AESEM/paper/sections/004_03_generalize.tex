\subsubsection{Generalizing to non-Euclidean Inputs}

This algorithm can be adapted to work on non-Euclidean inputs, however some heuristics used do not work outside a Euclidean or similarly restrictive input.

\paragraph{Center of Gravity}
In \cref{par:04:choosemodified} we used a center of gravity heuristic for $Rc_i$ that associated a point in space with the highest expected score,
by computing a weighted average of the nodes' coordinates weighted by their score.
This of course requires the points to have coordinates, so this would require the nodes to be situated in a coordinate space like the Euclidean space.
An option is to simply drop this heuristic and modify $W_i$ to say
\begin{align*}
	W_i & = 0.8 \cdot Rs_i + 0.2 \cdot Rp_i
\end{align*}
Here we removed the center of gravity heuristic and filled the now vacant space in the coefficients by increasing the coefficients for $Rs_i$ and $Rp_i$ by equal parts. 
This would only require the graph to be complete, since we need to calculate the distance between the currently visited node and every other unvisited node for $Rp_i$.

\paragraph{2-opt}
In \cref{subsubsec:04:createharmony} we use the 2-opt method \cite{croes_method_1958} to optimize the path in step 1 and 4 respectively.
In a non-Euclidean input it might be difficult to formulate a concept that would constitute as two edges \enquote{crossing}.
As such, the path improvement can be reduced to consist only of step 2 and 3.
These should work on any graph since they do not make assumptions about the graph's properties.

\paragraph{Navigation}
Similarly to the S-Algorithm (see \cref{subsec:03:salgo}) this algorithm navigates through the graph by adding new nodes until only the end node is available.
With the same argumentation as in \cref{par:03:salgotriangle} it can be said that the Szwarc-Boryczka algorithm likely will deliver worse results on a graph that does not satisfy the triangle equation.
Since navigation is a crucial part of these algorithms, this would likely strongly impact the results.

In summary, the algorithm should deliver decent results on graphs that satisfy the triangle equation.
However, to what extent this impacts the results remains to be tested empirically.
