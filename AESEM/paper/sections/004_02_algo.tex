\subsection{Algorithm}
\label{subsec:04:algo}

\citeauthor{szwarc_novel_2022} applied the concepts explained in \cref{subsec:04:hs} to the OP. \cite{szwarc_novel_2022}

The solution vectors in $HM$ are arrays that store the nodes visited by each path in the order in which they are visited.
%We will use the Euclidean graph in \cref{fig:04:graphexample} as an example.
%For simplicity, we omit the bound $T_{max}$ in this example.
A possible state of an $HM$ is depicted in \cref{fig:04:hm}.

\iffalse
\begin{figure}
	\centering
	\begin{tikzpicture}
		\node at (0,0) (1) {$v_1$};
		\node[left=-3mm of 1] {\tiny{$(0,0)$}};
		\node at (1,1) (2) {$v_2$};
		\node[above left=-3mm of 2] {\tiny{$(1,1)$}};
		\node[above right=-3mm of 2] {\small{$2$}};
		\node at (1,-1) (3) {$v_3$};
		\node[below left=-3mm of 3] {\tiny{$(1,-1)$}};
		\node[below right=-3mm of 3] {\small{$5$}};
		\node at (3,1) (4) {$v_4$};
		\node[above left=-3mm of 4] {\tiny{$(3,1)$}};
		\node[above right=-3mm of 4] {\small{$2$}};
		\node at (3,-1) (5) {$v_5$};
		\node[below left=-3mm of 5] {\tiny{$(3,-1)$}};
		\node[below right=-3mm of 5] {\small{$5$}};
		\node at (4,0) (6) {$v_6$};
		\node[right=-3mm of 6] {\tiny{$(4,0)$}};

		\draw (1) -- (2);
		\draw (1) -- (3);
		\draw (1) -- (4);
		\draw (1) -- (5);
		\draw (1) -- (6);
		\draw (2) -- (3);
		\draw (2) -- (4);
		\draw (2) -- (5);
		\draw (2) -- (6);
		\draw (3) -- (4);
		\draw (3) -- (5);
		\draw (3) -- (6);
		\draw (4) -- (5);
		\draw (4) -- (6);
		\draw (5) -- (6);
	\end{tikzpicture}
	\caption{A Euclidean graph with a position and score for each node.}
	\label{fig:04:graphexample}
\end{figure}
\fi

\begin{wrapfigure}{l}{0.5\textwidth}
	\centering
	\adjustbox{width=0.45\textwidth, keepaspectratio}{
		\begin{tikzpicture}[every node/.style={draw,minimum width=1cm,minimum height=1cm}]
			\matrix [row sep=3mm] (mat) {
				\node[draw=none] (0) {$0$}; & \node {$v_1$}; & \node{$v_3$}; & \node{$v_2$}; & \node {$v_5$}; & \node {$v_6$}; \\
				\node[draw=none] {$1$}; & \node {$v_1$}; & \node {$v_4$}; & \node {$v_3$}; & \node{$v_6$};                           \\
				\node[draw=none] {$2$}; & \node {$v_1$}; & \node{$v_2$}; & \node{$v_4$}; & \node{$v_5$}; & \node{$v_6$}; \\
			};
			\node[draw=none, above=0cm of 0] {$i$};
			\node[draw=none, above=0cm of mat] {$HM$};
		\end{tikzpicture}%
	}%
	\caption{An example $HM$ with paths starting at node $v_1$ and ending at node $v_6$.}
	\label{fig:04:hm}
\end{wrapfigure}

This results in the solution vectors not being the same size.
This is not an issue, but it might raise the question of how values for a new harmony, that is a new path, are chosen.

\subsubsection{Choosing the next Node}

Of course the path always starts at the start node, in the case of the example in \cref{fig:04:hm}, it is node $v_1$.
If at some point in the algorithm, the path so far is $P$ (a list of nodes) which node should we pick next?
As described in \cref{subsec:04:hs}, we have three possibilities.

\paragraph{Choosing from $HM$}

We search for the occurrences of the currently visited node $c \in P$ in $HM$ and keep track of the nodes that appear \emph{after} $c$ in $HM$.
These nodes appearing after $c$ are inserted into a tentative list $L$ of candidates that could become the next node.
From $L$ we need to remove all nodes that already appeared in $P$ and also remove all nodes that would not allow travelling to the end node due to violating $T_{max}$.
We then assign each node in $L$ the objective function value of the harmony they come from and choose a node randomly using a weighted distribution based on the assigned values.

It could be the case that $L$ is empty. The nodes previously in $L$ could have violated $T_{max}$ or $c$ did not appear in $HM$.
In this case, we insert all unvisited nodes that do not violate $T_{max}$ into $L$ and weigh a node $v \in L$ by the quotient of its score by the distance between $c$ and $v$, so: $\tfrac{s(v)}{t(c, v)}$.
We then choose a node again with a weighted distribution based on these weights.

\phantomsection
\label{par:04:choosemodified}
\paragraph{Choosing a modified value}

Let $U \subset V$ be the set of unvisited nodes.
To modify we use three heuristics and rank all nodes by each of them.

\begin{enumerate}
	\item $Rs_i$ ranks the node $v_i$ by the score it contains. The node with the highest score is rank $1$ and so on.
	\item $Rc_i$ ranks the node $v_i$ with coordinates $(x_i, y_i)$ by its closeness to the center of gravity in terms of score. The center of gravity is calculated as follows:
	      \begin{equation*}
		      x := \frac{\sum_{v_i \in U} x_i \cdot s(v_i)}{\sum_{v_i \in U} s(v_i)}\quad\quad
		      y := \frac{\sum_{v_i \in U} y_i \cdot s(v_i)}{\sum_{v_i \in U} s(v_i)}
	      \end{equation*}
	      The node closest to the center of gravity is rank $1$ and so on.
	\item $Rp_i$ ranks the node $v_i$ by its distance to the last visited node.
	      The closest node is rank $1$ and so on.
\end{enumerate}

Each node's weight is then defined as:
\begin{equation*}
  W_i := 0.7 \cdot Rs_i + 0.2 \cdot Rc_i + 0.1 \cdot Rp_i
\end{equation*}

We then choose a node from $U$ with each node being weighted by its $W_i$.
It should be noted that we prioritize \emph{lower} $W_i$.

\paragraph{Choosing a random Node}

As the title suggests, we choose a random node from the ones which do not violate $T_{max}$.

\subsubsection{Creating a new Harmony}
\label{subsubsec:04:createharmony}

After repeated choices no other node except the end node can be chosen, we complete the new harmony $P$ by inserting the end node.
Then we calculate its objective function value and check whether it is better than the worst harmony stored in $HM$.
If it is not better, the solution is discarded and a new iteration begins.
If it \emph{is} better, we improve $P$ in four steps.

\begin{enumerate}
  \item We optimize the path using the \emph{2-opt} method which was first described by \citeauthor{croes_method_1958}. \cite{croes_method_1958}
    For each pair of edges $\{a,b\}, \{c, d\} \in E$ that are part of $P$ we check whether they cross each other. 
    If so, we reorder the nodes in $P$ such that $\{a, c\}$ and $\{b, d\}$ are now part of the path instead. Now they do not cross anymore.
  \item We remove the node $v_i \in P$ with the lowest value of the following quotient:
    \begin{equation*}
      \frac{s(v_i)}{t(v_j, v_i) + t(v_i, v_k) - t(v_j, v_k)}
    \end{equation*}
    Where $v_j, v_k \in P$ are the preceeding and succeeding nodes of $v_i$ in $P$ respectively.
    This ratio is a measure of how great the return of this node is in relation to the additional distance needed to travel to it.
  \item We add the unvisited node $v_i$ with the highest value of the aforementioned quotient. 
    This means that we need to check each node for every position it could be inserted in $P$.
  \item If $P$ was improved in the previous step, apply 2-opt to it again.
\end{enumerate}

The resulting path $P$ now replaces the worst harmony in $HM$.

\subsubsection{Avoiding premature Convergence}

It might be the case that after some amount of iterations $HM$ contains only very similar paths.
To remedy this, if some $R \in \mathbb{N}$ iterations passed without the worst harmony in $HM$ being replaced,
all harmonies in $HM$ except the best one are discarded and filled with random paths again.

