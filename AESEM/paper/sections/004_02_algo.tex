\subsection{Algorithm}
\label{subsec:04:algo}

\citeauthor{szwarc_novel_2022} applied the concepts explained in \cref{subsec:04:hs} to the \op{} \cite{szwarc_novel_2022}.

The solution vectors in \hm{} are arrays that store the nodes visited by each path in the order in which they are visited.
%We will use the Euclidean graph in \cref{fig:04:graphexample} as an example.
%For simplicity, we omit the bound $T_{max}$ in this example.
A possible state of an \hm{} is depicted in \cref{fig:04:hm}.

\begin{figure}
	\centering
	\begin{tikzpicture}[nd/.style={draw,shape=circle}]
		\tikzset{cross/.style={cross out, draw,
					minimum size=2*(#1-\pgflinewidth),
					inner sep=0pt, outer sep=0pt},
			cross/.default={3pt}};

		\node[nd] at (0,0) (1) {$v_1$};
		\node[below=-1mm of 1] {\tiny{$(0,0)$}};

		\node[nd] at (1,1) (2) {$v_2$};
		\node[below=-1mm of 2] {\tiny{$(1,1)$}};
		\node[above=-1mm of 2] {\small{$2$}};

		\node[blue, nd] at (1,-1) (3) {$v_3$};
		\node[below=-1mm of 3] {\tiny{$(1,-1)$}};
		\node[above=-1mm of 3] {\small{$5$}};

		\node[blue, nd] at (2,2) (4) {$v_4$};
		\node[below=-1mm of 4] {\tiny{$(2,2)$}};
		\node[above=-1mm of 4] {\small{$5$}};

		\node[blue, nd] at (2,-2) (5) {$v_5$};
		\node[below=-1mm of 5] {\tiny{$(2,-2)$}};
		\node[above=-1mm of 5] {\small{$2$}};

		\node[nd] at (3,1) (6) {$v_6$};
		\node[below=-1mm of 6] {\tiny{$(3,1)$}};
		\node[above=-1mm of 6] {\small{$2$}};

		\node[red, nd] at (3,-1) (7) {$v_7$};
		\node[below=-1mm of 7] {\tiny{$(3,-1)$}};
		\node[above=-1mm of 7] {\small{$5$}};

		\node[nd] at (4,0) (8) {$v_8$};
		\node[below=-1mm of 8] {\tiny{$(4,0)$}};

		\draw[->] (1) -- (2);
		\draw[->] (2) -- (7);
		\draw[blue,->] (7) -- (4);
		\draw[blue,->] (7) -- (3);
		\draw[blue,->] (7) -- (5);

		\node[cross,orange] at (2,0) (aa) {};
		\draw[orange] (2,0) circle (3pt);

	\end{tikzpicture}
	\caption{A Euclidean graph with a position and score for each node. The algorithm is building a new path. The red node is currently the last node of the path. The blue edges and nodes are considered by the algorithm to be inserted next.
		The center of gravity is marked in orange. It is only coincidental that it is in the center of the graph.}
	\label{fig:04:graphexample}
\end{figure}

\begin{wrapfigure}{l}{0.5\textwidth}
	\centering
	\adjustbox{width=0.45\textwidth, keepaspectratio}{
		\begin{tikzpicture}[every node/.style={draw,minimum width=1cm,minimum height=1cm}]
			\matrix [row sep=3mm] (mat) {
				\node[draw=none] (0) {$0$}; & \node {$v_1$}; & \node[red]{$v_7$}; & \node[blue]{$v_5$}; & \node{$v_8$};                           \\
				\node[draw=none] {$1$}; & \node {$v_1$}; & \node {$v_3$}; & \node[red]{$v_7$}; & \node[blue]{$v_4$}; & \node{$v_8$}; \\
				\node[draw=none] {$2$}; & \node {$v_1$}; & \node{$v_2$}; & \node[red]{$v_7$}; & \node[blue]{$v_3$}; & \node{$v_8$}; \\
			};
			\node[draw=none, above=0cm of 0] {$i$};
			\node[draw=none, above=0cm of mat] {\hm{}};
		\end{tikzpicture}%
	}%
	\caption{An example \hm{} with paths starting at node $v_1$ and ending at node $v_8$.}
	\label{fig:04:hm}
\end{wrapfigure}

This results in the solution vectors possibly not being the same size.
This is not an issue, but it might raise the question of how values for a new harmony, that is a new path, are chosen.

\subsubsection{Choosing the next node}

The path always begins at the start node. In the case of the example in \cref{fig:04:hm}, it is node $v_1$.
If at some point in the algorithm, the path so far is $P$ (a list of nodes), as described in \cref{subsec:04:hs}, we have three possibilities to pick the next node.

\paragraph{Choosing from HM}

We search for the occurrences of the currently visited node $c \in P$ in \hm{} and keep track of the nodes that appear \emph{after} $c$ in \hm{}.
These nodes appearing after $c$ are inserted into a tentative list $L$ of candidates that could become the next node.
From $L$ we need to remove all nodes that already appeared in $P$ and also remove all nodes that would not allow travelling to the end node due to violating $T_{max}$.
We then assign each node in $L$ the objective function value of the harmony they come from and choose a node randomly using a weighted distribution based on the assigned values.

It could be the case that $L$ is empty. The nodes previously in $L$ could have violated $T_{max}$ or $c$ did not appear in \hm{}.
In this case, we insert all unvisited nodes that do not violate $T_{max}$ into $L$ and weigh a node $v \in L$ by the quotient of its score by the distance between $c$ and $v$, so: $\tfrac{s(v)}{t(c, v)}$.
We then choose a node again with a weighted distribution based on these weights.

In the example in \cref{fig:04:graphexample,fig:04:hm} we have $P = [v_1, v_2, v_7]$.
We search for $v_7$ in \hm{} and find that it is followed by $v_3$, $v_4$ and $v_5$ respectively.
So $L = \{v_3, v_4, v_5\}$. Their respective weights are $\tfrac{5}{2}, \tfrac{4}{\sqrt{10}}$ and $\tfrac{2}{\sqrt{2}}$.
We now choose one of these nodes by weighing them by those weights.

\phantomsection
\label{par:04:choosemodified}
\paragraph{Choosing a modified value}

Let $U \subset V$ be the set of unvisited nodes.
To modify we use three heuristics and rank all nodes by each of them.

\begin{enumerate}
	\item $Rs_i$ ranks the node $v_i$ by the score it contains. The node with the highest score is rank $1$ and so on.
	\item $Rc_i$ ranks the node $v_i$ with coordinates $(x_i, y_i)$ by its closeness to the center of gravity in terms of score. The center of gravity is calculated as follows:
	      \begin{equation*}
		      x := \frac{\sum_{v_i \in U} x_i \cdot s(v_i)}{\sum_{v_i \in U} s(v_i)}\quad\quad
		      y := \frac{\sum_{v_i \in U} y_i \cdot s(v_i)}{\sum_{v_i \in U} s(v_i)}
	      \end{equation*}
	      The node closest to the center of gravity is rank $1$ and so on.
	\item $Rp_i$ ranks the node $v_i$ by its distance to the last visited node.
	      The closest node is rank $1$ and so on.
\end{enumerate}

Each node's weight is then defined as:
\begin{equation*}
	W_i := 0.7 \cdot Rs_i + 0.2 \cdot Rc_i + 0.1 \cdot Rp_i
\end{equation*}

The coefficients are also adopted from \citeauthor{szwarc_novel_2022} \cite{szwarc_novel_2022}.
They prioritize a node's score while its closeness to the center of gravity and its distance from the last visited node have less of an impact.
We then choose a node from $U$ with each node being weighted by its $W_i$.
It should be noted that we prioritize \emph{lower} $W_i$.

\begin{table}
	\centering
	\begin{tabular}[c]{|c|c|c|c|c|c|}
		\hline
		\multirow{2}{*}{\textbf{Node}} & \multicolumn{3}{c|}{\textbf{Ranks}} & \multirow{2}{*}{$W_i$}                  \\
		                               & $Rs_i$                              & $Rc_i$                 & $Rp_i$ &       \\\hline
		$v_3$                          & $1$                                 & $1$                    & $2$    & $1.1$ \\\hline
		$v_4$                          & $1$                                 & $2$                    & $3$    & $1.4$ \\\hline
		$v_5$                          & $2$                                 & $2$                    & $1$    & $1.9$ \\\hline
		$v_6$                          & $2$                                 & $1$                    & $2$    & $1.8$ \\\hline
	\end{tabular}
	\caption{The rank values for the example in \cref{fig:04:graphexample,fig:04:hm}.}
	\label{tab:04:ranks}
\end{table}

To calculate the values for our example, we first calculate the center of gravity:

\begin{align*}
	x := \frac{42}{21} = 2 & y := \frac{0}{21} = 0
\end{align*}

The values for the examples in \cref{fig:04:graphexample,fig:04:hm} are depicted in \cref{tab:04:ranks}.
Using the ranks as weights, prioritizing lower weights, we choose one of the four nodes.

\paragraph{Choosing a random node}

As the title suggests, we choose a random node from the ones which do not violate $T_{max}$.
After repeated choices no other node except the end node can be chosen, we complete the new harmony $P$ by inserting the end node.

\subsubsection{Improving the Harmony}
\label{subsubsec:04:createharmony}

After calculating the path $P$ we calculate its objective function value and check whether it is better than the worst harmony stored in \hm{}.
If it is not better, the solution is discarded and a new iteration begins.
If it \emph{is} better, we improve $P$ in four steps.

\begin{enumerate}
	\item We optimize the path using the \emph{2-opt} method which was first described by \citeauthor{croes_method_1958} \cite{croes_method_1958}.
	      For each pair of edges $\{a,b\}, \{c, d\} \in E$ that are part of $P$ we check whether they cross each other.
	      If so, we reorder the nodes in $P$ such that $\{a, c\}$ and $\{b, d\}$ replace the original two edges. Now they do not cross anymore. In practice, one would choose an upper limit of iterations, to limit the time this requires.
	\item We remove the node $v_i \in P$ with the lowest value of the following quotient:
	      \begin{equation*}
		      \frac{s(v_i)}{t(v_j, v_i) + t(v_i, v_k) - t(v_j, v_k)}
	      \end{equation*}
	      Where $v_j, v_k \in P$ are the preceding and succeeding nodes of $v_i$ in $P$ respectively.
	      This value is a measure of how great the return of this node is in relation to the additional distance needed to travel to it.
	\item We add the unvisited node $v_i$ with the highest value of the aforementioned ratio.
	      This means that we need to check each node for every position it could be inserted in $P$.
	\item If $P$ was improved in the previous step, apply 2-opt to it again.
\end{enumerate}

The resulting path $P$ now replaces the worst harmony in \hm{}.

\subsubsection{Avoiding premature Convergence}

It might be the case that after some amount of iterations \hm{} contains only very similar paths.
If that is the case, then each iteration of the algorithm might not improve the best solution, since there is only little variety to pull from when creating a new harmony.
Therefore, we should avoid having the entirety of \hm{} prematurely converging to a possibly only local optimum.
To remedy this, if some $R \in \mathbb{N}$ iterations passed without the worst harmony in \hm{} being replaced,
all harmonies in \hm{} except the best one are discarded and filled with random paths again.
This avoids the algorithm stagnating because of all paths in \hm{} being too similar to one another.
