\subsection{Deterministic Algorithm}
\label{subsec:03:dalgo}

The second algorithm presented is the \emph{Deterministic Algorithm} (D-Algorithm).
This will only be briefly outlined for reasons explained later.

\begin{wrapfigure}{l}{0.5\textwidth}
    \centering
    \begin{tikzpicture}[pienode/.style 2 args={ circle, minimum size=#1, inner sep=0pt, path picture={\fill[red!30] (path picture
    bounding box.center) -- ++(0:#1) arc[start angle=0, end angle=3.6*#2,
    radius=#1] --cycle;}}]
        \node[pienode={4cm}{60}] (a) {};
        \node[circle,fill=white,minimum size=2cm] {};

        \node[draw,shape=circle,minimum width=4cm] at (0,0) (a) {};
        \node[draw,shape=circle,minimum width=2cm] at (0,0) (a) {};

        \draw (0,0) -- (0:2);
        \draw (0,0) -- (216:2);
        \draw[black!40] (0.3,0) arc[start angle=0, end angle=216, radius=3mm];

        \draw[-Circle] (170:1.2) -- (200:1.9);
        \draw[Circle-] (170:1.2) -- (160:1.7);
        \draw[-Circle] (160:1.7) -- (130:1.1);
        \draw[Circle-] (100:1.8) -- (130:1.1);
        \draw[-Circle] (100:1.8) -- (70:1.3);
        \draw[Circle-] (50:1.5) -- (70:1.3);
        \draw[-Circle] (50:1.5) -- (10:1.8);
    \end{tikzpicture}
    \caption{An example of the D-Algorithm creating a path restricted to a section.}
    \label{fig:03:dalgoexample}
\end{wrapfigure}

The D-Algorithm is inspired by \citeauthor{wren_computer_1972} and their algorithm for vehicle routing \cite{wren_computer_1972}.
Space is divided into different sections each given using concentric circles with different radii and an arc each.
An example of this is depicted in \cref{fig:03:dalgoexample}.
In \citeauthor{tsiligiridis_heuristic_1984} case, twelve different radii and arcs of $90\degree$ are used.
So space is divided into four quadrants, each being divided into $13$ segments.

A route is successively created for each sector, each route only using nodes that are in the respective sector.
This is done to reduce the travel distance from node to node.
In each sector, nodes are added until either no other node can be added due to violating $T_{max}$ or if all nodes of this sector have already been visited.

\subsubsection{Generalizing to non-Euclidean Inputs}
\label{subsubsec:03:dalgogeneralize}

As mentioned before, this algorithm is only briefly explained.
The reason for this is that this algorithm cannot be easily generalized to non-Euclidean inputs.

\paragraph{Space}
This algorithm divides \emph{space} into sectors 
and using this terminology already hints at the fact that we need to measure and quantify \emph{space}, for example by using a metric.
And this leads us back to a Euclidean input.

\phantomsection
\label{par:03:dalgogeneralizemetrics}
\paragraph{Other Metrics} 
Of course, the Euclidean metric is by far not the only one in existence.
One could possibly construct a similar division of the space by using a different metric, if a replacement for the concept of arcs and angles can be formulated.

Though, even under that assumption, this would shift the original problem to requiring another metric instead of the Euclidean one, 
in addition to finding adaptations and replacements for aspects of the algorithm that do not work when not using the Euclidean metric.
In addition, especially for real-world applications the Euclidean metric is by far the most prevalent and useful (see \cref{subsec:02:reasons}).
So not only would we be replacing the one problem with another, we would most likely be requiring an even rarer input due to requiring a more niche metric.
Needless to say, this would not be a generalization of this algorithm.

\paragraph{Not using any Metrics}
If not using a metric at all, one would need to partition the graph without the ability to use this metric. 
One would need to do away with the idea of sectioning space into sectors and instead divide the graph in another way.
This could perhaps be done using algorithms that allow dividing the graph into clusters, like Girvan and Newman's algorithm \cite{girvan_community_2002} which tries to identify edges connecting \emph{tightly-knit groups} and isolating those groups.
Though one would have to evaluate the quality of the solution with experiments.

In summary, there seems to be no trivial generalization of this algorithm to less restricted inputs.
