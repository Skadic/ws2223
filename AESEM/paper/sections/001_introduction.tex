\section{Introduction}
\label{sec:01:introduction}

The amount of logistic problems that are concerned with routing are ever increasing.
To use the metaphor of the Traveling Salesman Problem and borrow from \citeauthor{chao_fast_1996} et al. \cite{chao_fast_1996},
say there is a salesman, who is en route from a certain city to another city and wants to visit customers in different locations while on the way.
He might not have enough time to visit every customer, so he seeks a route which maximizes the profit while respecting his time restriction.

Such problems can be modeled by the \enquote{Orienteering Problem} (OP), or variants thereof,
whose basic premise is the collection of score or profit from specific locations while on a time or travel length budget. 
This problem was described by \citeauthor{tsiligiridis_heuristic_1984} et al. \cite{tsiligiridis_heuristic_1984} who related it to the eponymous orienteering sport.
In this sport, both a start and end point are given. Participants now have the task of navigating to designated checkpoints which will award them with points upon reaching them.
While doing so, they are subject to a time limit which if exceeded, will lead to disqualification or penalization.

As the OP is an NP-hard problem \cite{golden_orienteering_1987},
efficient but possibly not exact algorithms are of interest.

We will introduce some algorithms from the literature, older ones as well as a recent one 
and then discuss their applicability to general formulations of the orienteering problem. 
\missing{Actually maybe describe what I want to do}
