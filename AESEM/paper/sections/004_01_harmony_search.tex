\subsection{Harmony Search}
\label{subsec:04:hs}

\emph{Harmony Search} \cite{geem_optimal_2000,geem_state---art_2010} is an optimization technique developed by \Citeauthor{geem_optimal_2000}.
It makes use of an analogy where Jazz musicians, during their improvisation sessions, try to find pitches that produce aesthetically pleasant harmonies by each playing a pitch on their instrument. \cite{geem_introduction_2022}
There is a vast amount of possible harmonies, but only comparatively few would be aesthetically pleasing.
The musicians, at first, might not know which pitches to play in order to create a pleasant-sounding harmony.
But the more the ensemble plays together, the more each player will remember which pitches, when played, lead to a good result.
With this knowledge they can make more well-informed decisions, possibly leading to even better results.
While making a decision, they might decide to stick with pitches that worked well in the past or might try to vary those pitches by a small amount.

It is this analogy that describes the way the harmony search technique operates.
As with the aforementioned example, solutions to optimization problems in computer science can be considered to be vectors in a search space.
These vectors correspond to harmonies in the analogy and are also called \emph{harmony} here.
Thus, each decision variable of this vector corresponds to a musician and each value produced by the variable corresponds to a pitch played by said musician.
Of course in optimization, we are not searching for \enquote{aesthetically pleasing} solution vectors, but rather we have some measure of quality for any solution vector, an objective function.
We keep a list of solution vectors $HM$ (Harmony Memory) of size $HMS \in \mathbb{N}$ (Harmony Memory Size), which represents the memory of the musicians.
In this list, we keep the best harmonies we have found so far but in the beginning, this will be filled with random solution vectors.

In each iteration we create a new harmony, check if the new harmony is better than the worst harmony in $HM$ and if so, replace the worst harmony in $HM$ with it.
We continue iterating until either the best solution has reached a desired quality according to the objective function, or until a set number of iteration has elapsed.

\subsubsection{Creating a new Harmony}

As for how a new harmony is created, let $HMCR \in [0,1]$ (Harmony Memory Consideration Rate) be the rate by which each decision variable chooses a value from $HM$ as its value in the new harmony.

If a value is chosen from $HM$, let $PAR \in [0,1]$ be the rate at which the value chosen from $HM$ will be slightly modified.
For example if some decision variable's allowed values were the integers $\{1, \dots, 10\}$ and the value $4$ was chosen from $HM$, it might be modified such that the resulting value is a number adjacent to $4$, so $3$ or $5$.

If a value is not chosen from $HM$ it is uniformly randomly chosen from the set of allowed values for this decision variable.

This results in the three ways of choosing a value, shown in \cref{tab:04:hschoices}.

\begin{table}[t]
    \centering
    \begin{tabular}{|c|c|}\hline
        \textbf{Probability} & \textbf{Choice}\\ \hline
        $HMCR \cdot (1 - PAR)$ & Value from $HM$ without modification\\\hline
        $HMCR \cdot PAR$ & Value from $HM$ and modify it\\\hline
        $1 - HMCR$ & Random allowed value\\\hline 
    \end{tabular}
    \caption{The choices available when creating a new harmony.}
    \label{tab:04:hschoices}
\end{table}
