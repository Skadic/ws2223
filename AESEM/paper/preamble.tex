\nonstopmode

% Basic Packages
%\usepackage[utf8]{inputenc}
%\usepackage[T1]{fontenc}
\usepackage[USenglish]{babel}

% Geometry
\usepackage[a4paper,
            bindingoffset=0.2in,
            left=1.5in,
            right=1.5in,
            top=1in,
            bottom=1in,
            footskip=.3in]{geometry}
            
% Math and Symbols
\usepackage{amsmath}
\usepackage{amssymb}

% Textstuff
\usepackage{csquotes}
\usepackage{url}
\usepackage{hyperref}
\usepackage{lmodern}            % Provides the Latin Modern Font which offers more glyphs than the default Computer Modern
\usepackage[nameinlink]{cleveref}
\crefname{figure}{figure}{figures}
%\crefname{subfigure}{Abbildung}{Abbildung}
%\crefname{table}{Tabelle}{Tabelle}
%\crefname{listing}{Quelltext}{Quelltext}
%\crefname{chapter}{Kapitel}{Kapitel}
%\crefname{section}{Abschnitt}{Abschnitt}
%\crefname{subsection}{Abschnitt}{Abschnitt}
%\crefname{subsubsection}{Abschnitt}{Abschnitt}
%\crefname{beispiel}{Beispiel}{Beispiel}
%\crefname{lemma}{Lemma}{Lemma}

% Text Writing Stuff
\usepackage{todonotes}
\usepackage{xargs}
% https://tex.stackexchange.com/questions/9796/how-to-add-todo-notes
\newcommandx{\unsure}[2][1=]{\todo[linecolor=red,backgroundcolor=red!25,bordercolor=red,#1]{#2}}
\newcommandx{\change}[2][1=]{\todo[linecolor=blue,backgroundcolor=blue!25,bordercolor=blue,#1]{#2}}
\newcommandx{\info}[2][1=]{\todo[linecolor=green,backgroundcolor=green!25,bordercolor=green,#1]{#2}}
\newcommandx{\missing}[2][1=]{\todo[linecolor=yellow!25,backgroundcolor=yellow!25,bordercolor=yellow,#1]{#2}}
\newcommandx{\improvement}[2][1=]{\todo[linecolor=Plum,backgroundcolor=Plum!25,bordercolor=Plum,#1]{#2}}
\newcommandx{\thiswillnotshow}[2][1=]{\todo[disable,#1]{#2}}

% Set Paragraph Skip
%\setlength{\parskip}{0.5\baselineskip}%
%\setlength{\parindent}{0pt}%

% gfx
\usepackage{pgfpages}
\usepackage{svg}
\usepackage{graphicx}
\usepackage{xcolor}
\usepackage{color}
\usepackage{wrapfig}
\usepackage{subcaption}

% Tikz
\usepackage{tikz}
\usetikzlibrary{positioning,calc,shapes,decorations}

% Bibliography
\usepackage[maxcitenames=1]{biblatex}
\addbibresource{Orienteering.bib}


