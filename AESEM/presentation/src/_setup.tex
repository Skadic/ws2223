\documentclass[aspectratio=169,10pt]{beamer}
%\includeonlyframes{current}
\nonstopmode

% Basic Packages
%\usepackage[utf8]{inputenc}
%\usepackage[T1]{fontenc}
\usepackage[USenglish]{babel}

% Math and Symbols
\usepackage{amsmath}
\DeclareMathOperator*{\argmax}{arg\,max}
\DeclareMathOperator*{\argmin}{arg\,min}
\usepackage{amssymb}
%\usepackage{gensymb}
%\newtheorem{definition}{Definition}
%\newtheorem{theorem}{Theorem}

% Beamer
\usetheme[progressbar=frametitle]{metropolis}
\newcommand{\themename}{\textbf{\textsc{metropolis}}\xspace}
\setbeameroption{show notes on second screen}

% Itemized notes please
\makeatletter
\def\beamer@setupnote{%
  \gdef\beamer@notesactions{%
    \beamer@outsideframenote{%
      \beamer@atbeginnote%
      \beamer@notes%
      \ifx\beamer@noteitems\@empty\else
      \begin{itemize}\itemsep=0pt\parskip=0pt%
        \beamer@noteitems%
      \end{itemize}%
      \fi%
      \beamer@atendnote%
    }%
    \gdef\beamer@notesactions{}%
  }
}
\makeatother

% Textstuff
\usepackage{csquotes}
%\usepackage{url}
%\usepackage[normalem]{ulem}
%\usepackage{multirow}
%\usepackage{algorithm}
%\usepackage{algpseudocode}
%\usepackage{algorithmicx}
%\usepackage{lmodern}            % Provides the Latin Modern Font which offers more glyphs than the default Computer Modern
%\usepackage{hyperref}
%\usepackage[nameinlink]{cleveref}
%\crefname{figure}{figure}{figures}
%\crefname{equation}{equation}{equations}
%\crefname{subfigure}{Abbildung}{Abbildung}
%\crefname{table}{Tabelle}{Tabelle}
%\crefname{listing}{Quelltext}{Quelltext}
%\crefname{chapter}{Kapitel}{Kapitel}
%\crefname{section}{Abschnitt}{Abschnitt}
%\crefname{subsection}{Abschnitt}{Abschnitt}
%\crefname{subsubsection}{Abschnitt}{Abschnitt}
%\crefname{beispiel}{Beispiel}{Beispiel}
%\crefname{lemma}{Lemma}{Lemma}

% Text Writing Stuff
%\usepackage{todonotes}
%\usepackage{xargs}
% https://tex.stackexchange.com/questions/9796/how-to-add-todo-notes
%\newcommandx{\unsure}[2][1=]{\todo[linecolor=red,backgroundcolor=red!25,bordercolor=red,#1]{#2}}
%\newcommandx{\change}[2][1=]{\todo[linecolor=blue,backgroundcolor=blue!25,bordercolor=blue,#1]{#2}}
%\newcommandx{\info}[2][1=]{\todo[linecolor=green,backgroundcolor=green!25,bordercolor=green,#1]{#2}}
%\newcommandx{\missing}[2][1=]{\todo[linecolor=yellow!25,backgroundcolor=yellow!25,bordercolor=yellow,#1]{#2}}
%\newcommandx{\improvement}[2][1=]{\todo[linecolor=Plum,backgroundcolor=Plum!25,bordercolor=Plum,#1]{#2}}
%\newcommandx{\thiswillnotshow}[2][1=]{\todo[disable,#1]{#2}}

\newcommand{\tsplong}{\textsc{TravellingSalesmanProblem}}
\newcommand{\tsp}{\textsc{TSP}}
\newcommand{\oplong}{\textsc{OrienteeringProblem}}
\newcommand{\op}{\textsc{OP}}

% Set Paragraph Skip
%\setlength{\parskip}{0.5\baselineskip}%
%\setlength{\parindent}{0pt}%

% gfx
%\usepackage{pgfpages}
%\usepackage{svg}
\usepackage{graphicx}
%\usepackage{xcolor}
%\usepackage{color}
%\usepackage{wrapfig}
%\usepackage{subcaption}
%\usepackage{adjustbox}

% Tikz
\usepackage{tikz}
\usepackage{calc}
%\usepackage{siunitx}
\usetikzlibrary{positioning,calc,shapes,decorations,arrows.meta}

% Bibliography
\usepackage[maxcitenames=1]{biblatex}
\addbibresource{Orienteering.bib}

\title{Generalization of Algorithms for the Orienteering Problem}
\author{Etienne Palanga}
\date{\today}
